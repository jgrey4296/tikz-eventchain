%% compiler:lualatex
%% available compilers : lualatex pdflatex xelatex
% !TeX TS-program = lualatex
% !TeX encoding = UTF-8
\documentclass[class=article, crop=false]{standalone}

%%-- imports
\usepackage{iftex}
\usepackage{fontspec}
\usepackage[colorlinks=true, linkcolor=blue, urlcolor=blue, pdfstartview=FitH]{hyperref}
\usepackage[numbers]{natbib}
\usepackage[russian,english]{babel}
\usepackage[version=3]{mhchem}
\usepackage{doi}
\usepackage{listings}
\usepackage{url}
\usepackage{lettrine}
\usepackage{titling}
\usepackage{pdflscape}
\usepackage{geometry}
%%-- end imports

\usepackage{responsive-title}
\usepackage[debug]{eventchain}

\def\eventChainTitle{Event Chains Example}
\title{\eventChainTitle}
\author{John Grey}
\date{21-09-2025}
\setlength{\parskip}{2em}

\tikzset{eventchain/.style={font=\tiny}}

\begin{document}
\resptitle{\eventChainTitle}{\eventChainTitle}

Examples of the use of `eventchain'. Both vertical and horizontal. 

\begin{figure}
  \begin{eventchain}[start=5, dist=0.5cm]
    \state
    \begin{event}[name=First Event, dist=1cm, cols=2, fmt={c | l}]
      bloo ($ \alpha \omega $) \\
    \end{event}
    \begin{fluents}[name=Fluents, dist=2cm, cols=2, fmt={c | l}]
      \addfluent{blah} \\
      \subfluent{bloo($ \alpha \omega $)} \\
      \keepfluent{aweg} \\
    \end{fluents}
    \begin{event}[name=Second Event, cols=2, fmt={c | l}]
      \subfluent{aweg} \\
      awegjoi \\
    \end{event}
    \begin{event}[name=Third Event]
      baweg \\
      awegjoi \\
    \end{event}
  \end{eventchain}
  \caption{A Top-Bottom example}
\end{figure}

\pagebreak

\begin{landscape}
  \begin{figure}[th]
    \caption{A Left-Right Example}
    \begin{eventchain}[dir=right, dist=0.2cm]
      \state[5]
      \state
      \state
      \state
      \state
      % \event
      \begin{event}[name=Event, dist=0.7cm]
        blah \\
        aweg \\
      \end{event}
      \state
      \begin{fluents}[name=State, dist=0.7cm, cols=2, fmt={c | l}]
        \addfluent{blah} \\
        \subfluent{bloo($ \alpha \omega $)} \\
      \end{fluents}
      \state
      \state
      \state
      \jumpto[100]
    \end{eventchain}
  \end{figure}

  \hrulefill

  \begin{figure}[bh]
    \begin{eventchain}[dir=right, dist=0.2cm]
      \state[5]
      \state
      \state
      \jumpto[10]
      \state
      \state
      % \event
      \begin{event}[name=Event, dist=0.7cm]
        blah \\
        bloo($ \alpha \omega $) \\
      \end{event}
      \state
      \begin{fluents}[name=State, dist=0.7cm, cols=2, fmt={c | l}]
        \addfluent{blah} \\
        \subfluent{bloo($ \alpha \omega $)} \\
      \end{fluents}
      \state
      \state
      \state
      \jumpto[100]
    \end{eventchain}
    \caption{A Left-Right Example}
  \end{figure}
\end{landscape}
% \restoregeometry

% Pellentesque dapibus suscipit ligula. Donec posuere augue in quam. Etiam vel
% tortor sodales tellus ultricies commodo. Suspendisse potenti. Aenean in sem ac
% leo mollis blandit. Donec neque quam, dignissim in, mollis nec, sagittis eu,
% wisi. Phasellus lacus. Etiam laoreet quam sed arcu. Phasellus at dui in ligula
% mollis ultricies. Integer placerat tristique nisl. Praesent augue. Fusce
% commodo. Vestibulum convallis, lorem a tempus semper, dui dui euismod elit,
% vitae placerat urna tortor vitae lacus. Nullam libero mauris, consequat quis,
% varius et, dictum id, arcu. Mauris mollis tincidunt felis. Aliquam feugiat
% tellus ut neque. Nulla facilisis, risus a rhoncus fermentum, tellus tellus
% lacinia purus, et dictum nunc justo sit amet elit.



% \nocite{*}
% \bibliographystyle{jg_custom}
% \bibliography{}
\end{document}
